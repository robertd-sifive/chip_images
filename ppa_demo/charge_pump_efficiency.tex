\documentclass[aspectratio=169]{beamer}
\usetheme{metropolis}

\usepackage{amsmath}
\usepackage{mathdots}
\usepackage{mathtools}
\usepackage{amssymb}
\usepackage{xcolor}
\usepackage{tipa}
\usefonttheme[onlymath]{serif}

\definecolor{qorvoblue}{HTML}{009EE1}
\definecolor{qorvoblue2}{HTML}{0076A9}
\definecolor{qorvogrey}{HTML}{262626}

\setbeamercolor{progress bar}{fg=qorvogrey}
\setbeamercolor{title separator}{fg=qorvogrey}
\setbeamercolor{progress bar in head/foot}{fg=qorvogrey}
\setbeamercolor{progress bar in section page}{fg=qorvogrey}

\title{AD2004C GaN PA Controller IC}
\subtitle{\textbf{Charge pump design}\\
\textcolor[HTML]{009EE1}{\textbf{Doc \# = NA}}\\
\textcolor[HTML]{009EE1}{\textbf{REV = 0}}}
\author{Robert D'Angelo}
%\institute{Qorvo}
\date{\today}

\usebackgroundtemplate%
{%
    \includegraphics[width=\paperwidth,height=\paperheight]{./figures/qorvo_presentation_background_alpha.pdf}%
}

\defbeamertemplate{frame numbering}{altcounter}{
\vspace{-2.5mm}
$|$\insertframenumber
\hspace{7mm}
}
\setbeamertemplate{frame numbering}[altcounter]
\setbeamercolor{frame numbering}{fg=white}

\newcommand\myfootnote[1]{%
  \tikz[remember picture,overlay]
  \draw (current page.south west) +(1in + \oddsidemargin,2em)
  node[anchor=south west,inner sep=0pt]{\parbox{\textwidth}{%
      \rlap{\rule{10em}{0.4pt}}\raggedright\tiny#1}};
}


% for schematics graphics
\newcommand{\trimmedgraphic}[1]{%
    %\centering
    \centerline{
        \makebox[\textwidth]{
                    \includegraphics[trim = 5cm 3cm 3cm 8.5cm,clip,width=0.9\linewidth]{#1}
                }
    }
}

\newcommand{\trimmedgraphicconv}[2]{
    \begin{figure}
        \centering
        \includegraphics[trim = 2cm 0cm 2cm 0cm,clip,width=0.33\linewidth]{#1}
        \caption*{#2}
    \end{figure}
}

% place graphic in the center with max size on the slide
\newcommand{\centeredgraphic}[1]{
    \centerline{
        \makebox[\textwidth]{
                    \includegraphics[width=\linewidth,height=0.8\textheight, keepaspectratio]{#1}
                }
    }
}

% place graphic in the center with max size on the slide
\newcommand{\centeredgraphicsmall}[1]{
    \centerline{
        \makebox[\textwidth]{
                    \includegraphics[height=0.5\textheight, keepaspectratio]{#1}
                }
    }
}


\newcommand{\centeredgraphictwo}[2]{
    \centerline{
        \makebox[\textwidth]{
                    \includegraphics[width=0.5\linewidth,height=0.8\textheight, keepaspectratio]{#1}
                    \includegraphics[width=0.5\linewidth,height=0.8\textheight, keepaspectratio]{#2}
                }
    }
}
\newcommand{\centeredgraphictwovert}[2]{
    \centerline{
        \makebox[\textwidth]{
                \begin{tabular}{c}
                    \includegraphics[height=0.4\textheight, keepaspectratio]{#1} \\
                    \includegraphics[height=0.4\textheight, keepaspectratio]{#2}
                \end{tabular}
                }
    }
}



% place graphic in the center with max size on the slide with transparency
\newcommand{\centeredgraphictransparent}[1]{
    \centerline{
        \makebox[\textwidth]{
                    \includegraphicstransparent[width=\linewidth,height=0.8\textheight, keepaspectratio]{#1}{15}
                }
    }
}

\newcommand{\centeredgraphicwidthtransparent}[2]{
    \centerline{
        \makebox[\textwidth]{
                    \includegraphicstransparent[width=#2\linewidth,height=0.8\textheight, keepaspectratio]{#1}{15}
                }
    }
}

% place graphic in the center with given width
\newcommand{\centeredgraphicwidth}[2]{
    \centerline{
        \makebox[\textwidth]{
                    \includegraphics[width=#2\linewidth,height=0.8\textheight, keepaspectratio]{#1}
                }
    }
}

% place graphic in the center with height and width specified, but aspect ratio maintained
\newcommand{\centeredgraphicaspect}[3]{
    \centerline{
        \makebox[\textwidth]{
                    \includegraphics[width=#2\linewidth,height=#3\textheight, keepaspectratio]{#1}
                }
    }
}

\newcommand{\centeredgraphicaspectfootnote}[4]{
    \centerline{
        \makebox[\textwidth]{
                    \includegraphics[width=#2\linewidth,height=#3\textheight, keepaspectratio]{#1}
                }
    }
    \myfootnote{#4}
}

% place a frame that is just an image and a title
\newcommand{\imageonlyframetwo}[4]{% image, image, title, width
\begin{frame}
    \frametitle{#3}
    \pause
    \begin{overprint}
    \onslide<1>\centeredgraphicwidth{#1}{#4}
    \onslide<2>\centeredgraphicwidthtransparent{#1}{#4}
    \end{overprint}

    \begin{overprint}
    \onslide<1>\centeredgraphicwidthtransparent{#2}{#4}
    \onslide<2>\centeredgraphicwidth{#2}{#4}
    \end{overprint}
\end{frame}
}

% place a frame that is just an image and a title
\newcommand{\imageonlyframe}[3]{% image, title, caption
\begin{frame}
    \frametitle{#2}
    #3
    \visible<1>{\centeredgraphic{#1}}
\end{frame}
}

\newcommand{\imageonlyframenonum}[3]{% image, title, caption
\begin{frame}[noframenumbering]
    \frametitle{#2}
    #3
    \visible<1>{\centeredgraphic{#1}}
\end{frame}
}



\newcommand{\imageonlyframetest}[4]{% image, title, caption
\begin{frame}
    \frametitle{#3 \includegraphics[width=0.05\linewidth, keepaspectratio]{#2}}
    #4
    \visible<1>{\centeredgraphic{#1}}
\end{frame}
}


% place a frame that is just an image and a title
\newcommand{\imageonlyframenopause}[3]{% image, title, caption
\begin{frame}
    \frametitle{#2}
    #3
    \centeredgraphic{#1}
\end{frame}
}


%\newcommand\blfootnote[1]{%
  %\begingroup
  %\renewcommand\thefootnote{}\footnote{#1}%
  %\addtocounter{footnote}{-1}%
  %\endgroup
%}

%\newcommand{\myfootnote}[1]{
    %\setbeamerfont{footnote}{size=\tiny}
    %\blfootnote{#1}
%}

% place a frame that is just an image and a title and a footnote
\newcommand{\imageonlyframefootnote}[4]{% image, title, footnote, caption
\begin{frame}
    \frametitle{#2}
    #4
    \vspace{-0.5cm}
    \centeredgraphic{#1}
    \myfootnote{#3}
\end{frame}
}

\newcommand{\imageonlyframefootnotenoskip}[4]{% image, title, footnote, caption
\begin{frame}
    \frametitle{#2}
    #4
    \centeredgraphic{#1}
    \myfootnote{#3}
\end{frame}
}

\newcommand{\imageonlyframefootnotenoskipnonum}[4]{% image, title, footnote, caption
\begin{frame}[noframenumbering]
    \frametitle{#2}
    #4
    \centeredgraphic{#1}
    \myfootnote{#3}
\end{frame}
}



\newcommand{\imageonlyframefootnoteaspect}[6]{% image, title, footnote, caption, width, height
\begin{frame}
    \frametitle{#2}
    #4
    \centeredgraphicaspect{#1}{#5}{#6}
    \myfootnote{#3}
\end{frame}
}






\begin{document}

%\begin{frame}
    %\titlepage{}
%\end{frame}

%\begin{frame}
    %\frametitle{Overview}
    %\tableofcontents{}

    %\section{Analog computation}
        %\subsection{Analog multipliers}
        %\subsection{Time-mode arithmetic circuits}
    %\subsection{Scalable analog circuits}
        %\subsection{Time-mode circuis}
        %\subsection{Bulk driven circuits}
    %\section{Analog to information converters}
        %\subsection{Compressed sensing ADCs}
        %\subsection{Machine learning ADCs}
    %\section{Biomedical signals}
        %\subsection{ExG: EEG, ECG, EOG, etc.}
%\end{frame}

\bgroup
\makeatletter
\setbeamertemplate{footline}
{
  \leavevmode%
  \hbox{%
  %
  %
  }%
  \vskip0pt%
}
\makeatother

\setbeamercolor{normal text}{fg=white}
\setbeamercolor{frametitle}{fg=white}
\usebeamercolor[fg]{normal text}
{
\usebackgroundtemplate{\includegraphics[width=\paperwidth]{./figures/qorvo_presentation_title_page.pdf}}%

\begin{frame}
    \titlepage{}
    %\centerline{\includegraphics[scale=0.5]{figures/source_mixing_cichocki2005.pdf}}
\end{frame}
}
\egroup

\setcounter{framenumber}{0}

\setbeamercolor{normal text}{fg=black}
\setbeamercolor{frametitle}{fg=qorvoblue2, bg=qorvogrey}
%\usebeamercolor[fg]{normal text}
%\begin{frame}
    %\frametitle{Charge pump topologies}
    %\begin{itemize}
    %\item Dickinson
        %\begin{itemize}
        %\item on chip caps
        %\item body effect issues
        %\end{itemize}
    %\end{itemize}
    %%\centerline{\includegraphics[scale=0.5]{figures/source_mixing_cichocki2005.pdf}}
%\end{frame}


\imageonlyframe{./figures/gan_pa.pdf} % figure
               {GaN PA System}             % title
               {} % caption

\begin{frame}
    \frametitle{Charge pump system}
    %\vspace{2mm}
    \centeredgraphictwovert{figures/charge_pump_overview.pdf}{figures/charge_pump_timing.png}
    %\begin{align}
       %V_{n} = \frac{C V_{cc}}{C + C_s} - V_{t}\left[V_{\scriptscriptstyle SB}^{n}\right]
    %\end{align}

    %\myfootnote{Pan, F., \& Samaddar, T. (2010). Charge pump circuit design. McGraw Hill Professional.}
\end{frame}

\begin{frame}
    \frametitle{Charge pump ripple}
    \centeredgraphic{figures/cp_ripple.pdf}{}{}
    \vspace{-1cm}
    \begin{align}
        \Delta V_{pp} = \frac{I_{out}}{k_{fb} C_{load} f_{osc}} \\
        k_{fb} \approx 10
    \end{align}

    %\myfootnote{Pan, F., \& Samaddar, T. (2010). Charge pump circuit design. McGraw Hill Professional.}
\end{frame}



\begin{frame}
    \frametitle{Dickinson vs. Cascaded Doubler (i.e. $2^N$)}
    %\vspace{2mm}
    \centeredgraphic{figures/bootstrapped_cp_stage.pdf}
    %\begin{align}
       %V_{n} = \frac{C V_{cc}}{C + C_s} - V_{t}\left[V_{\scriptscriptstyle SB}^{n}\right]
    %\end{align}

    %\myfootnote{Pan, F., \& Samaddar, T. (2010). Charge pump circuit design. McGraw Hill Professional.}
\end{frame}

\begin{frame}
    \frametitle{Dickinson vs. Cascaded Doubler (i.e. $2^N$)}
    %\vspace{2mm}
    \centeredgraphic{figures/charge_pump_doubler_switches.pdf}
    %\begin{align}
       %V_{n} = \frac{C V_{cc}}{C + C_s} - V_{t}\left[V_{\scriptscriptstyle SB}^{n}\right]
    %\end{align}

    %\myfootnote{Pan, F., \& Samaddar, T. (2010). Charge pump circuit design. McGraw Hill Professional.}
\end{frame}



\begin{frame}{Pros and Cons}
\begin{columns}[t]
    \begin{column}{0.5\linewidth}
        \textbf{Dickinson} \\
        \begin{column}{0.5\linewidth}
            Pros:
            \begin{itemize}
                \item Half the parasitics
                \item Can use extra external cap footprints for larger load cap
            \end{itemize}
        \end{column}
        \begin{column}{0.5\linewidth}
            Cons:
            \begin{itemize}
                \item Saturation $\rightarrow$ $\uparrow$ $\frac{V_{loss}}{stage}$
                \item Parasitics have larger effect in saturation ($2/3 C_{ox}WL$)
                \item Bias bulks, bootstrap the gates for oxide reliability
            \end{itemize}
        \end{column}
    \end{column}
    \begin{column}{0.5\linewidth}
        \textbf{Doubler} \\
        \begin{column}{0.5\linewidth}
            Pros:
            \begin{itemize}
                \item Oxide stress met by architecture, no special bulk biasing needed
                \item Parasitics have smaller effect in triode ($1/2 C_{ox}WL$)
                \item $f_{ripple}$ effectively halved
            \end{itemize}
        \end{column}
        \begin{column}{0.5\linewidth}
            Cons:
            \begin{itemize}
                \item Double the parasitics
                \item Need double the footprints for external capacitors
            \end{itemize}
        \end{column}
    \end{column}
\end{columns}    
\end{frame}

\begin{frame}
    \frametitle{Charge pump design}
        \begin{itemize}
            \item $N_{\scriptscriptstyle stages} = \frac{V_{\scriptscriptstyle out} - V_{\scriptscriptstyle th}}{V_{\scriptscriptstyle clk} - V_{\scriptscriptstyle th}}$
            \item Spectre $\rightarrow$ $\mu_{p}$, $\gamma_{p}$ (body effect factor), $C_{ox}$, $V_{th}$
            \item Voltage, Temp rating, $\rightarrow$ $C_{\scriptscriptstyle ext} = 220nF$
            \item $f_{pump_{max}} = 50MHz$
            \item $T_{reco} = 1\mu \rightarrow f_{pump_{min}} \approx 1.5MHz$
            \item $L = 700nm$ minimum length to minimize parasitic cap
            \item $I_{out}, \Delta V \rightarrow W = 50\mu \times 16 \times 8$
            \item $\Delta V < \frac{V_{gate_{fs}}}{2 ^ {ENOB}}$, $ENOB > 9$
            \item Regulation options: discrete clock driver switching, VCO feedback, PWM feedback
        \end{itemize}
    %\centerline{\includegraphics[scale=0.5]{figures/source_mixing_cichocki2005.pdf}}
\end{frame}

\begin{frame}
    \frametitle{Stage driver parasitics}
        \begin{itemize}
            \item $\frac{C_{par_{drv}}}{C_{boost}} = \frac{V_{out} f_{max} L^2}{\mu_{p} N_{stages} (V_{dd} ^ 2 - V_{dd} V_{th} - \frac{V_{dd}}{2})}$
            \item $C_{par_{int}}$ more complex
            \item Guess: $C_{par_{int}} \approx 0.1$ for Dickinson
            \item $0.15$ for Doubler
        \end{itemize}
    %\centerline{\includegraphics[scale=0.5]{figures/source_mixing_cichocki2005.pdf}}
\end{frame}

\begin{frame}
    \frametitle{Ripple estimation}
        \begin{itemize}
            \item $C_{prl} = C_{boost} || C_{load}$
            \item $R_{sw} = \frac{(V_{clk} - V_{th}) L}{\mu_{p} C_{ox} W}$
            \item $\Delta V = \frac{I_{out}}{f_{osc}(C_{load} + C_{boost})} - \frac{R_{sw} I_{out} C_{load} C_{boost}}{(C_{load} + C_{boost})^2} \left(1 + ln(\frac{1}{f_{osc} R_{sw} C_{prl}})\right)$
        \end{itemize}
    %\centerline{\includegraphics[scale=0.5]{figures/source_mixing_cichocki2005.pdf}}
\end{frame}

\begin{frame}
    \frametitle{Efficiency estimation}
    \begin{align}
        \eta = \frac{V_{out} I_{out}}{V_{out}I_{out} + \frac{N_{stages} I_{out}^2}{f_{pump} C_{boost}} + (C_{par_{int}} + C_{par_{drv}}) N_{stages} f_{pump} C_{boost} V_{clk}^2}\nonumber
    \end{align}
    %\centerline{\includegraphics[scale=0.5]{figures/source_mixing_cichocki2005.pdf}}
\end{frame}

\begin{frame}
    \frametitle{$\eta$ and $\Delta V$ vs. $I_{out}$ and $f_{pump}$}
    %\vspace{2mm}
    \centeredgraphic{./figures/charge_pump_efficiency_comparison.pdf}
    %\begin{align}
       %V_{n} = \frac{C V_{cc}}{C + C_s} - V_{t}\left[V_{\scriptscriptstyle SB}^{n}\right]
    %\end{align}

    %\myfootnote{Pan, F., \& Samaddar, T. (2010). Charge pump circuit design. McGraw Hill Professional.}
\end{frame}

\begin{frame}
    \frametitle{$\eta$ and $\Delta V$ vs. $I_{out}$ and $C_{boost}$}
    %\vspace{2mm}
    \centeredgraphic{./figures/charge_pump_efficiency_comparison_cap.pdf}
    %\begin{align}
       %V_{n} = \frac{C V_{cc}}{C + C_s} - V_{t}\left[V_{\scriptscriptstyle SB}^{n}\right]
    %\end{align}

    %\myfootnote{Pan, F., \& Samaddar, T. (2010). Charge pump circuit design. McGraw Hill Professional.}
\end{frame}


\end{document}
